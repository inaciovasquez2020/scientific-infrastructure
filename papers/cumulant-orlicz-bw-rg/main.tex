\documentclass[11pt,a4paper]{article}

\usepackage[margin=1in]{geometry}
\usepackage{amsmath,amssymb,amsthm,mathtools}
\usepackage{bm}
\usepackage{hyperref}

\title{\textbf{Cumulant Hopf Structures, Orlicz \(\Gamma_2\) Transport,
BW Curvature Spectrum, mod-\(\phi\) Flow Rigidity, and Exact RG via Birkhoff Factorization}}
\author{Inacio F.\ Vasquez}
\date{}

\newtheorem{theorem}{Theorem}
\newtheorem{definition}{Definition}
\newtheorem{proposition}{Proposition}
\newtheorem{lemma}{Lemma}
\newtheorem{corollary}{Corollary}

\begin{document}
\maketitle

\section{Hopf algebra of cumulant diagrams}

Let \(\mathcal H=\bigoplus_{n\ge0}\mathcal H_n\) be a connected graded Hopf algebra over \(\mathbb R\) generated as an algebra by connected cumulant diagrams, with multiplication \(G_1G_2:=G_1\sqcup G_2\) and unit \(1=\emptyset\).

\begin{definition}[Admissible subdiagram family]
Fix a class \(\mathsf{Adm}(G)\) of subdiagrams \(H\subseteq G\) such that
\[
\emptyset\in \mathsf{Adm}(G),\qquad G\in \mathsf{Adm}(G),
\]
and \(H\in\mathsf{Adm}(G)\Rightarrow H\) is a disjoint union of connected admissible components.
\end{definition}

Define coproduct
\[
\Delta G=\sum_{H\in\mathsf{Adm}(G)} H\otimes G/H.
\]
Counit \(\varepsilon(\emptyset)=1\), \(\varepsilon(G)=0\) for \(G\neq\emptyset\).
Antipode by recursion
\[
S(G)=-G-\sum_{\emptyset\subsetneq H\subsetneq G\atop H\in\mathsf{Adm}(G)} S(H)\,(G/H).
\]

\begin{lemma}[Hopf axioms reduction]
If \(\mathsf{Adm}\) is stable under contraction and disjoint union and satisfies the forest compatibility condition
\[
K\in\mathsf{Adm}(H),\,H\in\mathsf{Adm}(G)\ \Rightarrow\ K\in\mathsf{Adm}(G)\ \text{and}\ (G/H)/(H/K)\cong G/K,
\]
then \((\mathcal H,m,\Delta,\varepsilon,S)\) is a connected graded Hopf algebra.
\end{lemma}

Let \(G(\mathcal H)\) be the character group of \(\mathcal H\) valued in \(\mathbb R\), with convolution
\[
(\chi_1*\chi_2)(G)=\sum_{H\in\mathsf{Adm}(G)}\chi_1(H)\chi_2(G/H).
\]
Infinitesimal characters are linear maps \(\theta:\mathcal H\to\mathbb R\) with
\[
\theta(G_1G_2)=\theta(G_1)\varepsilon(G_2)+\varepsilon(G_1)\theta(G_2).
\]

\begin{lemma}[Pro-nilpotence criterion]
Assume \(\dim \mathcal H_n<\infty\) for all \(n\). Then the Lie algebra of infinitesimal characters with bracket
\[
[\theta_1,\theta_2]=\theta_1*\theta_2-\theta_2*\theta_1
\]
is pro-nilpotent, and \(\exp_*\) defines a bijection from infinitesimal characters to characters.
\end{lemma}

\begin{theorem}[Conditional: Lie algebra cohomology]
Assume the pro-nilpotence criterion and that the dual Hopf algebra \(\mathcal H^\vee=\bigoplus_n \mathcal H_n^*\) exists as a graded Hopf algebra. Then the continuous Chevalley--Eilenberg cohomology of the character Lie algebra satisfies
\[
H^\bullet_{\mathrm{cont}}(\mathfrak g,\mathbb R)\cong H^\bullet_{\mathrm{Hoch}}(\mathcal H^\vee,\mathbb R)
\]
under the standard Lie--Hochschild comparison map.
\end{theorem}

\section{Orlicz \(\Gamma_2\) and transport}

Let \((\Phi,\Phi^*)\) be a Young pair. On a smooth metric measure space with reference \(\pi\), define the Orlicz carré-du-champ
\[
\Gamma_\Phi(f)=\Phi^*(|\nabla f|),
\]
and assume an Orlicz curvature condition
\[
\Gamma_{2,\Phi}(f)\ge \kappa\,\Gamma_\Phi(f).
\]

\begin{definition}[Full Orlicz logarithmic Sobolev inequality]
We say \(\pi\) satisfies \(\mathrm{LSI}_\Phi(\kappa)\) if for all smooth \(g>0\),
\[
\mathrm{Ent}_\pi(g)\le \frac{1}{\kappa}\int g\,\Phi^*(|\nabla \log g|)\,d\pi.
\]
\end{definition}

Let \(W_\Phi\) denote the Orlicz transport cost induced by \(c_\Phi(x,y)=\Phi(d(x,y))\).

\begin{lemma}[Conditional: Orlicz Hamilton--Jacobi duality]
Assume existence of a Hamilton--Jacobi semigroup \(Q_t\) with
\[
\partial_t Q_t f + \Phi^*(|\nabla Q_t f|)=0
\]
in viscosity sense, and Kantorovich duality
\[
W_\Phi(\nu,\pi)=\sup_{f}\left\{\int f\,d\nu-\int Q_1 f\,d\pi\right\}.
\]
\end{lemma}

\begin{theorem}[Conditional: \(\mathrm{LSI}_\Phi \Rightarrow\) Orlicz transport]
Assume \(\mathrm{LSI}_\Phi(\kappa)\) and the Orlicz Hamilton--Jacobi duality lemma. Then there exists \(C=C(\kappa,\Phi)\) such that for all \(\nu\ll\pi\),
\[
W_\Phi(\nu,\pi)\le C\,\mathrm{KL}(\nu\|\pi).
\]
\end{theorem}

\begin{theorem}[Conditional: Orlicz transport \(\Rightarrow \mathrm{LSI}_\Phi\)]
Assume the dual formulation above, differentiability of \(t\mapsto \int e^{Q_t f}\,d\pi\), and an Orlicz hypercontractive estimate for \(Q_t\). Then \(\mathrm{LSI}_\Phi(\kappa')\) holds for some \(\kappa'>0\).
\end{theorem}

\begin{corollary}[Conditional: equivalence]
Under the combined assumptions of the two implications, full Orlicz LSI is equivalent to the Orlicz transport inequality.
\end{corollary}

\section{Bures--Wasserstein curvature spectrum and heat trace}

Let \(\mathcal P_d\) be the SPD manifold with BW metric
\[
g_P(H,K)=\mathrm{tr}(P^{-1}HP^{-1}K).
\]
At \(P=I\), identify \(T_I\mathcal P_d\cong \mathrm{Sym}_d\). For \(A\in\mathrm{Sym}_d\), define \(\mathcal C_A(B)=[A,B]\) on \(\mathfrak{gl}_d\).

\begin{proposition}[Commutator spectrum]
If \(A\) is diagonalizable with eigenvalues \(\lambda_1,\dots,\lambda_d\), then
\[
\sigma(\mathcal C_A)=\{\lambda_i-\lambda_j\}_{1\le i,j\le d}.
\]
\end{proposition}

\begin{theorem}[Curvature eigenvalues at \(I\)]
Assume the BW Levi-Civita connection agrees with the affine-invariant form at \(I\) on the \(\mathrm{O}(d)\)-reductive decomposition. Then sectional curvature eigenvalues associated to commutator modes are
\[
-\frac14(\lambda_i-\lambda_j)^2.
\]
\end{theorem}

\begin{theorem}[Conditional: heat trace asymptotics]
Assume a compact quotient or a confining potential perturbation yielding compact resolvent for \(\Delta_{BW}\). Then
\[
\mathrm{Tr}(e^{-t\Delta_{BW}})\sim \sum_{k\ge0} a_k\,t^{(k-d_{\mathrm{eff}})/2},
\]
with \(a_0=(4\pi)^{-d_{\mathrm{eff}}/2}\mathrm{Vol}(\mathcal P_d)\) and \(a_1\) determined by scalar curvature.
\end{theorem}

\section{mod-\(\phi\) preserving stochastic flows}

Let \(X_t\) be a Markov process with cumulant generating function \(\psi_t(s)=\log\mathbb E e^{sX_t}\).

\begin{definition}[Full mod-\(\phi\) invariance constraint]
A flow preserves full mod-\(\phi\) structure if there exist \(\alpha(t)>0\), \(\beta(t)\in\mathbb R\), \(\gamma(t)\in\mathbb R\) such that
\[
\psi_t(s)=\psi_0(\alpha(t)s)+\beta(t)s+\gamma(t)
\]
for all \(s\) in a neighborhood of \(0\).
\end{definition}

\begin{theorem}[Conditional: generator rigidity]
Assume \(\psi_t\) is \(C^2\) in \(t\) and analytic in \(s\) near \(0\), and \(X_t\) has a diffusion generator \(\mathcal L=b(x)\partial_x+\frac12 a(x)\partial_x^2\).
If full mod-\(\phi\) invariance holds for a non-degenerate \(\psi_0\), then \(a(x)\) is constant and \(b(x)\) is affine:
\[
a(x)\equiv \sigma^2,\qquad b(x)=ux+v,
\]
so \(\mathcal L= (ux+v)\partial_x+\frac{\sigma^2}{2}\partial_x^2\).
\end{theorem}

\section{Exact RG via Birkhoff factorization on characters}

Let \(\mathcal A\) be a commutative Rota--Baxter algebra with splitting \(\mathcal A=\mathcal A_-\oplus \mathcal A_+\). Let \(\chi(z)\in G(\mathcal H\otimes \mathcal A)\) be a loop of characters.

\begin{theorem}[Birkhoff decomposition]
If the grading on \(\mathcal H\) is compatible with \(\mathcal A\)-poles and the counterterm recursion converges degreewise, then there exist unique \(\chi_\pm(z)\) with \(\chi_-(z)\in G(\mathcal H\otimes \mathcal A_-)\), \(\chi_+(z)\in G(\mathcal H\otimes \mathcal A_+)\) such that
\[
\chi(z)=\chi_-(z)^{-1}\chi_+(z).
\]
\end{theorem}

Define the renormalized character \(\chi_R=\chi_+(0)\). Define the beta functional by logarithmic differentiation along the scale parameter \(\Lambda\):
\[
\beta=\left.\frac{d}{d\log\Lambda}\right|_{\Lambda=1}\chi_R.
\]

\begin{theorem}[Conditional: RG as Riemann--Hilbert problem]
Assume scale-locality implies holomorphicity of \(\chi\) on a punctured neighborhood and the splitting \(\mathcal A_\pm\) realizes boundary values. Then the RG flow on cumulants is equivalent to the Riemann--Hilbert factorization problem for \(\chi\), and \(\beta\) is determined by the residue of \(\chi_-\).
\end{theorem}

\end{document}
